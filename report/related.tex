Machine learning has been used as a means of genre classification since 1997 when Carnegie Mellon researchers ran classifications on five-second song snippets\cite{Dannenberg}.In 2002, Princeton graduate student George Tzanetakis compiled the GTZAN dataset in order to perform classifications using a greater set of musical features \cite{tzanetakis}. Other researchers have experimented with the dataset by attempting classification on using new feature sets, different classifiers, and other techniques. For instance, predictions have been made by training subclassifiers or different features and combining the results \cite{Hu}. Other methods include using image classification techniques on the spectrograms of the samples \cite{Wu}. According to Brundage, Gliner, Jin and Wolf, the highest published accuracy of GTZAN classification is 93\% \cite{sample}. In our approach, we wanted to see if we could achieve comparable results to these techniques using MFCC and other musical features extracted via the Matlab MIRToolbox\cite{MIR}. 